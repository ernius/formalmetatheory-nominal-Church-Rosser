\documentclass[utf,utf8x,hyperref=hidelinks,xcolor=table]{beamer} %handout,notes=only

% no margen
%\usepackage[margin=0.1in]{geometry}
\usepackage{graphicx}
%\usepackage[bw,references]{latex/agda}
%\usepackage[conor,references]{latex/agda}
\usepackage[references,links]{agda}
\usepackage{amsmath}
\usepackage{mathtools}
\usepackage{textgreek}
\usepackage{catchfilebetweentags}
\usepackage{tipa}
\usepackage{framed}
%\usepackage{biblatex}

%math
\newcommand{\alp}{\ensuremath{\alpha}}
\newcommand{\lamb}{\ensuremath{\lambda}}
\newcommand{\alphaeqsym}{\ensuremath{\sim_\alpha}}
\newcommand{\choice}{\ensuremath{\chi}}

%Agda
\newcommand{\freshin}[2]{\ensuremath{#1 \mathbin{\AgdaDatatype{\#}} #2}}
\newcommand{\freein}[2]{\ensuremath{#1 \mathbin{\AgdaDatatype{*}} #2}}
\newcommand{\lambAg}[2]{\ensuremath{\AgdaInductiveConstructor{ƛ}\, #1\, #2}}
\newcommand{\inAg}{\ensuremath{\mathbin{\AgdaFunction{∈}}}}
\newcommand{\ninAg}{\ensuremath{\mathbin{\AgdaFunction{∉}}}}
\newcommand{\neqAg}{\ensuremath{\mathbin{\AgdaInductiveConstructor{≢}}}}
\newcommand{\ap}[2]{#1 \ensuremath{\mathbin{\AgdaInductiveConstructorFunction{·}} #2}}
\newcommand{\var}[1]{\ensuremath{\AgdaInductiveConstructorFunction{v}\, #1}}
\newcommand{\fv}{\ensuremath{{\AgdaFunction{fv}}\,}}
\newcommand{\perm}{\ensuremath{\mathbin{\AgdaFunction{∙}}}}
\newcommand{\perma}{\ensuremath{\mathbin{\AgdaFunction{∙}_a}}}
\newcommand{\free}{\ensuremath{\mathbin{\AgdaFunction{*}}}}
\newcommand{\choiceAg}{\ensuremath{\AgdaFunction{χ}\,}}
\newcommand{\choiceAgaux}{\ensuremath{\AgdaFunction{χ'}\,}}
\newcommand{\alpeqAg}{\ensuremath{\mathbin{\AgdaDatatype{∼α}}}}
\newcommand{\swap}[3]{\ensuremath{(#1 \mathbin{\AgdaFunction{∙}} #2)\, #3}}

% \newcommand{\agdaf}[1]{\ensuremath{\AgdaFunction{#1}\,}}
% \newcommand{\agdaD}[1]{\ensuremath{\AgdaDatatype{#1}\,}}
% \newcommand{\agdav}[1]{\ensuremath{\AgdaBound{#1}\,}}

\DeclareUnicodeCharacter{411}{\textipa{\textcrlambda}}
\DeclareUnicodeCharacter{65288}{(}
\DeclareUnicodeCharacter{65289}{)}
\DeclareUnicodeCharacter{8788}{\ensuremath{\coloneqq}}
\DeclareUnicodeCharacter{8336}{\ensuremath{_a}}
\DeclareUnicodeCharacter{8799}{\ensuremath{\overset{?}{=}}}
\DeclareUnicodeCharacter{8759}{\ensuremath{\dblcolon}}
\DeclareUnicodeCharacter{8718}{\ensuremath{\square}}
\DeclareUnicodeCharacter{9657}{\ensuremath{\triangleright}}

%\usetheme{Berlin}
%\inserttocsection
%\mode<presentation>
%% \newcommand\Section[2][]{%
%%    \section<presentation>[#1]{#2}
%%    \section<article>{#2}
%% }

\addtobeamertemplate{frametitle}{\vspace*{-0.1cm}}{\vspace*{-0.5cm}}



\begin{document}

\title{Alpha-Structural Induction and Recursion for the Lambda Calculus in Constructive Type Theory}
\author{Ernesto Copello,\'Alvaro Tasistro,Nora Szasz, Ana Bove, Maribel Fern\'andez}

\date[LSFA ’15]{10th Workshop on Logical and Semantic Frameworks, with Applications.}

\begin{frame}
\titlepage
\end{frame}

%\begin{frame}
%\frametitle{Outline}
%\tableofcontents
%\end{frame}

\section{Motivation}

\begin{frame}{Motivation}
\begin{block}{}
  Studying and formalising reasoning techniques over programming languages.
  \begin{itemize}
  \item Reasoning like in pen-and-paper proofs.
  \item Using constructive type theory as proof assistant.
  \end{itemize}

  More specifically: $\lambda$-calculus \& Agda.
\end{block}

\pause

\bigskip

We study a formilisation of $\lambda$-calculus with the following characteristics:

\begin{itemize}
\item In its original syntax with only one sort of names, \\  like pen-and-paper.
\item Substitution  and \alp-conversion is based upon name swapping (\emph{Nominal} approach).
\end{itemize}

%%Hacerla mas sencilla, decir que es la sintaxis clasica con un solo tipo de variables, lo mas parecido a lapiz y papel, y recalcar que se basa en la operacion de swaping mas sencilla

%% \footnotesize{
%%   \begin{tabular}{c|c|c}
%%     Approach & Pros & Cons \\
%%     \hline
%%     One sort variables  & Simple one &
%%                           \begin{tabular}{c}
%%                             capture avoiding substitution \\
%%                             (Church 36 / Curry-Feys 58) \\
%%                             reasoning over $\alpha$-eq. classes (\emph{BVC}) \\
%%                           \end{tabular}  \onslide<3-> \\ 
%%     \hline

%%     Frege     & substitution & reasoning over $\alpha$-eq. classes \onslide<4->\\
%%     \hline

%%     De Bruijn & $\alpha$-eq. classes & 
%%                           \begin{tabular}{rl}
%%                                & indexes operations \\
%%                              + & well-formed predicates \\
%%                             \hline
%%                                & distance from intuitive proofs
%%                            \end{tabular}  \onslide<5->\\
%%     \hline

%%     Locally Nameless & less indexes operations & distance from intuitive proofs   \onslide<6->\\
%%     \hline

%%     HOAS      & substitution & has limitations   \onslide<7->\\
%%     \hline

%%     Nominal   & 
%%                 \begin{tabular}{c}
%%                   swapping \\
%%                   \cellcolor{blue!25} $\alpha$-eq. classes 
%%                 \end{tabular} 
%%              &
%%                 \begin{tabular}{c}
%%                introduce swapping lemmas \\
%%                choice axiom incompatible
%%                \end{tabular}
%%   \end{tabular}}
\end{frame}

\note[itemize]{

\item We are interested in studying reasoning techniques over programming languages.
\item Besides we want to formalise theses techniques in Constructive Type Theory.
\item Specifically, we choose as our object of study the $\lambda$-calculus
\item and Agda as proof assistant and programming language where we will develop some $\lambda$-calculus meta-theory.
\item But we do not want to diverge from classical pen-and-paper proofs.
\item So we study the most direct formalisation: the original syntax with one sort of variables.
\item We take the swapping operation from the nominal approach
\item to define \alp-conversion and substitution.
}

\section{Reasoning over $\alpha$-equivalence classes}

\begin{frame}{Reasoning over $\alpha$-equivalence classes} \smallskip

  \begin{block}{Barenregt's variable convention (BVC)}
    Each $\lambda$-term represents its $\alpha$ class, so we can assume that we have bound and free variables all different.
  \end{block}

  \bigskip
 \pause

  \begin{block}{A complete formilisation often implies:}
    \begin{itemize}
    \item Define a no capture substitution operation.
    \item Complete induction over the \emph{size} of terms is needed to fill the gap between terms and $\alpha$-equivalence classes.
    \item Prove that all properties being proved are preserved under $\alpha$-conversion ($\alpha$-compatibe  predicates).
    \end{itemize}
  \end{block}
\end{frame}

\note[itemize]{
\setlength\itemsep{1em}
\item Most of $lambda$ calculus's meta-theory implicitly work over \alp-equivalences classes of terms.
\item This common practise even has a name: Barendregt´s Variable convention, that says:
\item This practice often doesn´t make explicit:
  \begin{itemize}
  \item  A definition of substitution that avoids variable capture is needed
  \item  for some proofs an induction over the size of the terms must be made
  \item  And That the properties being proved should be preserved under $\alpha$-conversion:  
 \item   we call this $\alpha$-compatibility of the properties.
  \end{itemize}
}

\begin{frame}{Reasoning over $\alpha$-equivalence classes} 
  \begin{block}{We want to be able to reproduce the following induction sketch} \bigskip
     \underline{$\lambda$\ case:} \\
              \ \ To prove  $\forall x, M,\ P(M) \Rightarrow P(\lambda x M)$\ we  instead prove:
             \[ \begin{array}{c}
                    \exists A\subseteq V, A\ \text{finite},\forall x^*, M^*\ \text{renamings} / x^* \not\in A, \lambda x M \alphaeqsym \lambda x^* M^*, \\ 
                     P(M^*) \Rightarrow P(\lambda x^* M^*) 
                \end{array}  \]
  \end{block}

\end{frame}

\note[itemize]{
\setlength\itemsep{1em}
\item We want to reproduce the following induction sketch:
  \begin{itemize}
  \item instead of proving the classic abstraction case
  \item we can give some finite set of variables from where the abstraction variable will not be chosen
  \item and then for an \alp-equivalent renaming of the term with any variable not in the chosen context
  \item we prove the induction step.
  \end{itemize}
}

\section{Nominal approach}

\begin{frame}{Formalisation}

All the slides from now on are fragments of our compiled Agda code.

\end{frame}


\begin{frame}{Variable Swaping}

\ExecuteMetaData[Atom.tex]{swap} 

\end{frame}

\note[itemize]{
\setlength\itemsep{1.5em}
\item From this slide until the end of this presentation we will directly show Agda's code of our formalisation.
\item We begin with the basis of our formalization: which is the swapping operation.
\item Next we show the swapping operation over variables, which is a ternary funcion that takes two atoms to be swapped in a third one
\item and is defined by cases
}

\begin{frame}{Terms}
\medskip

\ExecuteMetaData[Term.tex]{term} 

\pause
\medskip
Swapping operation on terms is simpler than substitution, and with nicer properties than \emph{renaming}: \\

\begin{itemize}
\item Also changes bound variables $\Rightarrow$ no variable capture\\ 
$(y\ x) \bullet (\lambda\ x . x\ y) = \lambda\ y . y\ x $

\item Injective \\
      \ExecuteMetaData[Atom.tex]{injective} 

\item Idempotent \\
      \ExecuteMetaData[Atom.tex]{idempotent} \\
      $(y\ x) \bullet (\lambda\ y . y\ x) = \lambda\ x . x\ y $
\end{itemize}
\end{frame}

\note[itemize]{
\setlength\itemsep{1.5em}
\item Lambda terms are directly defined  as follows
\item The swapping operation is trivially  extended to $\lambda$-terms.
\item And we swap free variables, and also bound variables.
\item Swapping is simpler than substitution and with better properties than renaming:
  \begin{itemize}
  \item because it also changes bound variables, then there is no possible variable capture
  \item it is injective
  \item it is idempotent: sofor any swapping there always  exists the inverse operation !
  \end{itemize}
}

\begin{frame}{$\alpha$-conversion, not using substitution!}


\ExecuteMetaData[Alpha.tex]{alpha}

\medskip

Novel definition.

\begin{itemize}
\item syntax directed
\pause
\item equivalent to classical one
\pause
\item inspired in ``cofinite quantification''\\ (Brian Aydemir et al, ``Engineering formal metatheory'', 2008)
\end{itemize}



\end{frame}

\note[itemize]{
\setlength\itemsep{1em}
\item We present a syntax directed \alp-conversion definition.
\item 
  \begin{itemize}
    \item in the abstraction rule we have a stronger hypothesis than in the classic definition, that says:
    \item $\forall$\ variable $c$\ not in a given context $xs$, 
    \item we have the \alp-compatibility for the abstraction sub-terms$M,N$  with the binder variable swapped with $c$.
    \item it is Inspired in Aydemir's ``cofinite quantification'' 
  \end{itemize}
\item This definition fits with Barendregt´s convension, where we choose the bound variable different from some given finite context.
\item This definition is easier to use but more difficult to prove
\item But This is not a problem because \alp-compatibility is usually a hypothesis.
}

\section{$\alpha$-structural induction \& recursion}

\begin{frame}{Induction Principles for $\lambda$-terms}

In the following slides we will iterate over several induction principles, until we get one which captures Barenregt´s variable convention.

\bigskip

Each one can be derived from the former one.

\end{frame}


\begin{frame}{Primitive induction}

  Comes for free with the definition of terms : \bigskip

  \ExecuteMetaData[TermInduction.tex]{termPrimInduction}
\end{frame}

\note[itemize]{
\setlength\itemsep{1.5em}
\item This is the classic primitive induction principle where we have to prove the abstraction case for any  variable $b$.
\item But this induction principle comes for free with the simple syntax definition.
}

\begin{frame}{Permutation induction}

A permutation $\pi$ is a sequence of swappings.
 
\bigskip

\bigskip

We derive this induction principle, from simple structural induction:

 \bigskip

\ExecuteMetaData[TermInduction.tex]{termIndPermutation}





\end{frame}

\note[itemize]{
\setlength\itemsep{1.5em}
\item A permutation $\pi$ is a sequence of swappings.
\item We can derive the following induction principle for lambda terms
\item Stronger: Note thay the inductive hypothesis of the abstraction case is given for any permutation 
\item No need of induction on the length of terms!
\item Derived in the same way as we can derive the complete induction principle on natural numbers from primitive induction.  
}


\begin{frame}{$\alpha$-structural induction}


\ExecuteMetaData[Alpha.tex]{alphaCompatible}

\bigskip

We can derive this induction principle from the former one:

\ExecuteMetaData[TermInduction.tex]{alphaPrimInduction}

\bigskip

Now we can emulate the BVC!

\end{frame}

\note[itemize]{
\setlength\itemsep{1.5em}
\item We define what is to be an \alp-compatible predicate: that is, the predicate is preserved for {\bf \alp-equivalent} terms 
\item And then we obtain the next induction principle directly from the previous one and we get our goal.
\item Not that in the abstraction case of we can exclude  some finite set of variables 
}


\begin{frame}{$\alpha$-structural iteration}

\bigskip

  We can easily derive an iteration principle from the former principle.

\bigskip
\bigskip

\ExecuteMetaData[TermRecursion.tex]{termIteration}

%\begin{framed}
%  \begin{minipage}{0.15\linewidth}
%     Remember:
%  \end{minipage}
%  \begin{minipage}{0.6\linewidth}
%    \scriptsize{\ExecuteMetaData[TermInduction.tex]{alphaPrimInduction}}
%  \end{minipage}
%\vspace{-10px}
%\end{framed}

\end{frame}


\note[itemize]{
\setlength\itemsep{1.5em}
\item We also derive an iteration principle
\item In the abstraction case of this principle:
  \begin{itemize}
  \item we give some list of variables to be excluded 
  \item and a function that given a selected variable not in the  list (to be choosen as the abstraction variable)
  \item and the  result of the recursive call of the iteration
\item  constructs a result
  \end{itemize}
}


\begin{frame}{$\alpha$-structural iteration}

\ExecuteMetaData[Alpha.tex]{strongAlphaCompatible}      

\bigskip
\bigskip

\ExecuteMetaData[TermRecursion.tex]{iterationStrongCompatible}

\end{frame}

\note[itemize]{
\setlength\itemsep{1.5em}
\item Now we define what it means that a function is strong\alp-compatible: 
\item That is, for any two \alp-equivalent terms the result of the function is the same.
\item The following lema states that our iteration principle always defines strong\alp-compatible funcions
\item This is a direct consequence of the fact that the recursive call is done over a renamed sub-term
\item Note that because of this, the iteration principle has no way of extracting any information from bound variables
}



\begin{frame}{$\alpha$-structural recursion}
  Using the last iteration principle we define a recursion principle: 
  \bigskip

  \ExecuteMetaData[TermRecursion.tex]{termRecursion}

  \bigskip

  Inherits \alp-compatibility property from the iteration principle.
  
\end{frame}

\note[itemize]{
\setlength\itemsep{1.5em}
\item We generalize a recursion principle from the iterative one.
\item It inherits \alp-compatibility
}

\section{Applications}

%% \begin{frame}{Norrish}
%%   We successfully apply theses iteration/recursion principles to several difficult examples extracted from literature.
%% \end{frame}

\begin{frame}{Applications}

In the following slides we will show some classic results of the $\lambda$-calculus meta-theory that can be formalized with our principles.

\end{frame}



\begin{frame}{Free variables}

We use the iteration principle to define the free variables function. \bigskip

\ExecuteMetaData[FreeVariables.tex]{freeVariables} 

\bigskip

\bigskip

\ExecuteMetaData[FreeVariables.tex]{fvalpha} 

\end{frame}

\note[itemize]{
\setlength\itemsep{1.5em}
\item As an example we use the iteration principle to define the free variables function.
\item for the variable case we give the singleton list constructor
\item for the application we junt concatenate the recursive calls
\item for the abstraction case, given the abstraction variable and the recursive call, we just return the recursive call minus the  variable
\item The function is \alp-compatible by definition
}

\begin{frame}{Soundness of the fv function} \bigskip

\ExecuteMetaData[FreeVariables.tex]{fvPred} \smallskip

\small{\ExecuteMetaData[Term.tex]{free}}  \bigskip

\pause
We can define \AgdaDatatype{\_*\_} relation in the following equivalent way: \bigskip

\ExecuteMetaData[FreeVariables.tex]{free} \smallskip

Now \AgdaFunction{\_free\_}  is $\alpha$-compatible by definition, by being defined with our iteration principle.\\
As \AgdaFunction{fv} is strong \alp-compatible, then \AgdaFunction{Pfv*}\ is  $\alpha$-compatible!
\end{frame}

\note[itemize]{
\item We want to prove that the function we have just defined is sound.
\item First we define the property to be proven: Pfv* states that if a variable belongs to the result of fv applied to a term, then it is fee in that term
\item This is thee classical definition of  free variable
\item We can also use our iteration principle to define this predicate
\item For the variable case, it is a function that for a variable $b$\ it is returned a Set inhabited by a proof that $a \equiv b$.
\item The application case is the disjoint union of the recursive calls result sets.
\item Finally the abstraction case, as we can choose the binder variable not equal to $a$, so the set only inhabited by the proof that $a$ is free in the abstraction sub-term is directly a proof that $a$ is free in the abstraction.
\item And this predicate is $\alpha$-compatible by definition!!!
%\item We can return \AgdaDatatype{Set} in this example because of we use levels in our induction/recursion principles.
}

\begin{frame}{Soundness of the fv function} \bigskip

Abstraction case sketch of the \alp-structural proof. 

\[ \begin{array}{l}
  ih)\ \AgdaFunction{Pfv*}(M) \equiv\ a \in \fv\ M \Rightarrow \freein{a}{M} \\
  it)\ \AgdaFunction{Pfv*}(\lambAg{x}{M}) \equiv\ a \in \fv\ \lambAg{x}{M} \Rightarrow \freein{a}{\lambAg{x}{M}} \\ \\
  \underline{\text{Proof:}} \\
\\
  a \in \fv\ \lambAg{x}{M} \overset{iteration pr.}{\Rightarrow} a \in (\fv\ \swap{x}{b}{M}) - b,\text{with}\ b\ \text{fresh in}\ \lambAg{x}{M} \\ \\

  \text{By (-) operation prop.} \Rightarrow  b \neq a\ \wedge a \in \fv\ \swap{x}{b}{M}  \\ \\

   \text{We can restrict}\ x\ / x \neq a \overset{swap-* prop.}{\Rightarrow} a \in \fv M \overset{ih)}{\Rightarrow} \freein{a}{M} \overset{x \neq a}{\Rightarrow} \freein{a}{\lambAg{x}{M}} \\

   \end{array}
\]

\pause 

\end{frame}

\note[itemize]{
\setlength\itemsep{0.2em}
\item We can use our induction principle that mimics Barenregt´s convention convention because \AgdaDatatype{free} predicate is \alp-compatible.
\item In the abstration clase, which is the interesting one, note how our principle makes the proof very short and direct:
\item ei belongs to the free variables of lambda ex eM, then by being defined with the iteration principle,
\item ei belongs to the free variables of swapping ex with bi in eM
\item Now we can assume ex not equal to ei, and then by a swapping-aster property ei belongs to the free variables of eM,
\item finally, using again that ex is not not equal to ei, ei is free in lambda ex eM
\item The proof does not need to assume $x \neq ei$, it can be proved, but would require several more steps and properties in the proof.
}

\begin{frame}{Substitution}
\ExecuteMetaData[Substitution.tex]{substitution}

\bigskip

\pause

\ExecuteMetaData[Substitution.tex]{lemmaSubst1}
\end{frame}

\note[itemize]{
\setlength\itemsep{0.2em}
\item Next we define the substitution operation on terms using the iteration principle.
\item The variable case is defined with the hvar function, which replaces a variable by a term
\item The application case is just the application constructor applied to the recursive subtitution results.
\item In the abstraction case we can exclude the variable being susbtituted and the free variables of the term susbtituted to avoid any variable capture.
\item then we can proceed as in the application case, returning the abstraction constructor appied to the bound variable and the recursive substitution result.
\item This definition is \alp-compatible because it was defined with the iteration principle.
}


% \begin{frame}{Substitution} \bigskip

% We use \AgdaFunction{TermIndPerm} induction principle to prove: \medskip

% \ExecuteMetaData[Substitution.tex]{lemmaSubst2}

% \pause
% \bigskip
% With the two previous results we can now directly prove the next result. 

% \ExecuteMetaData[Substitution.tex]{lemmaSubst}
% \end{frame}

\begin{frame}{Substitution}

The substitution is alpha-equivalent to a na\"{\i}ve substitution when no variable capture exists. \bigskip

\small{\ExecuteMetaData[Substitution.tex]{naivesubstitution}} \\ 
\bigskip

\bigskip
\pause
With the previous result and the \alp-structural induction principle we can do any classic pen-and-paper proofs about substitution. \medskip

\pause
For example, the next classic result: \medskip

\small{\ExecuteMetaData[Substitution.tex]{Psubstcomp}}
\end{frame}

\note[itemize]{
\item The result of substitution is alpha-equivalent to a naive substitution when no variable capture exists. 
\item With the previous result and the \alp-structural induction principle we can prove any result about substitution and terms as in classic pen-and-paper proofs. 
\item For example, the next classic result about composition of substitutions for fresh enough variables  
}

\begin{frame}{Substitution} \bigskip

Abstraction case of the \alp-structural induction, choosing $b \not\in [y]\ +\!\!+\ fv\ N\ +\!\!+\ fv\ N [y := L] $  \smallskip

\scriptsize{\ExecuteMetaData[Substitution.tex]{abstractionComposition}}

\end{frame}

\note[itemize]{
\setlength\itemsep{1.5em}
\item In the abstraction case we can do the following direct equational proof, identical to a classical pen-and-paper proof, for a bi abstraction variable enough fresh.
}


\section{Results and Conclusions} 
\begin{frame}{Summary} \medskip


We present an induction principle that mimics Barenregt´s convention for $\alpha$-compatible predicates, which allows us to choose the bound name in the abstraction case so that it does not belong to a given list of names. \bigskip \bigskip

\pause

We derive a recursion principle which defines strong \alp-compatible functions. \bigskip
\pause

All results are derived from the first primitive recursion, and no induction on the length of terms or accessible predicates were needed. \bigskip \bigskip

\pause

With this basic framework we are able to reproduce classic pen-and-paper proofs in a formal proof assistant.
\end{frame}


\begin{frame}{Thank you!}
  Questions ? 
\end{frame}

\bibliographystyle{plain}
\bibliography{resumen}

\end{document}
